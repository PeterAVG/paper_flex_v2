I like the word monetize, but the way it's used may be confusing. Monetizing freezer flexibility is the process of making money out of it. So installing controls, having contracts with the aggregator, offering the service etc. What section II does is describing the steps towards making money. Characterizing flexibility, defining some bidding/control strategies etc. This is only the technical part of the monetization. Maybe you can think of a way to adjust the terminology.

It would be nice to replace the figures with tikz code in the end. Now it's probably easier to use a quick png version though :)

I was a little confused on the mFRR part. The average reader may not know the differences between reservation payment and activation payment and the rules change often, so it's good to give a clear overview.
All players can provide balancing power, but the reservation part is different and a player may choose to take part or not. Both DK1 and DK2 have balancing power but only DK1 (for now) has only an up-regulation reserve market.
Clarify why we care only for DK1?
Your notation implies that $p_{h}^{r,\uparrow}$ $\forall{h} = 0$ when not accepted, maybe state that?
Is it called regulating power bid or balancing? It's a bit confusing between the reserve and real-time activation parts. 
I think it's not correct to say that bids are activated based on the balancing price, since that price is unknown and is calculated ex-post, based on the most expensive activated bid.
Bids are activated in a merit-order fashion, as I remember.
I have some questions on (3) related to the penalty cost, rebound imbalance etc. 
Probably better discuss that with a call!